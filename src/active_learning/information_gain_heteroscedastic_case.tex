\begin{sectionbox}\nospacing
  So far we considered homoscedastic noise\cref{defn:homoscedastic_noise} but
  sometimes we may have heteroscedasctic\cref{defn:heteroscedastic_noise} noise
  $\std_{n}(x)$ $\iff$ different locations may have different noise i.e.\ to different sensors.\\
  \imp{Problem}: in the heteroscedas case the most uncertain outcomes are no longer necessarily the most informative.
\end{sectionbox}
\begin{corbox}\nospacing
  \begin{cor}[\hfill\proofref{proof:cor:mutal_information_maximization_heteroscedastic_gaussian}\newline
    Heteroscedastic Gaussian]\label{cor:mutal_information_maximization_heteroscedastic_gaussian}
    \begin{align}
      x_{t}&=\argmax_{x\in D}\frac{\text{epistemic uncertainty}}{\text{aleatoric uncertainty}}
             =\argmax_{x\in D}\frac{\std_{f}^2(x)}{\std_n^2(x)}
    \end{align}
    this can then be maximized.\\
    Let $A_{t}=\left\{x_{1},\ldots,x_{t}\right\}$ then it follows:
    \begin{align}
      \std^2_t(x)=\kernel(x,x)-\kernel_{x,A_t}\left(\Kernelm_{A_t,A_t}+\std^2\vec{I}\right)^{-1}\kernel_{x,A_t}
    \end{align}
  \end{cor}
\end{corbox}
%%% TeX-command-extra-options: "-shell-escape"
%%% Local Variables:
%%% mode: latex
%%% TeX-master: "../../formulary"
%%% End:
